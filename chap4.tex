\chapter{Análisis crítico}

El análisis de las propuestas destaca a HelenOS, SerenityOS y Visopsys como las fuentes de inspiración más relevantes para el proyecto. HelenOS se establece como la base principal por su diseño modular, pero SerenityOS y Visopsys aportan lecciones valiosas sobre cómo estructurar un sistema complejo debido a que contienen documentacion abundante (man pages) y lograr una interfaz gráfica funcional. A diferencia de los modelos puramente teóricos, estos sistemas muestran una arquitectura más cercana a la realidad. Sin embargo, esto conlleva un desafío: muchos de estos proyectos no tienen un enfoque estrictamente educativo, sino experimental o de desarrollo personal, lo que a veces prioriza la funcionalidad avanzada sobre la facilidad de aprendizaje para un estudiante universitario.

Un patrón común identificado es la fuerte dependencia de entornos tipo Unix (como Linux) para poder compilar y emular los sistemas operativos. La mayoría de las herramientas estándar y scripts de construcción están diseñados para funcionar en estas plataformas, lo que complica el acceso desde otros entornos. Además, se observó que la estructura de los proyectos varía mucho: mientras algunos buscan la simplicidad, otros como HelenOS requieren cadenas de herramientas complejas, lo que refleja la diferencia entre un sistema hecho para enseñar y uno hecho para investigar o innovar.

Finalmente, las dificultades técnicas fueron notables durante la fase de pruebas. El problema más recurrente fue la obtención de imágenes de disco (ISOs) listas para usar, lo que obligó a realizar compilaciones manuales del código fuente. Este proceso presentó obstáculos significativos, como los largos tiempos de compilación requeridos por sistemas grandes como HelenOS y la dificultad para configurar correctamente los emuladores. A esto se suma la escasez de documentación clara en varios proyectos, lo que hizo que entender cómo ejecutar o corregir errores en el sistema fuera una tarea compleja y, en ocasiones, frustrante para un entorno de aprendizaje.Se espera poder construir un sistema operativo(implementar) teniendo como base estos sistemas ya vistos en el presente informe.