\chapter{Análisis crítico}

\section*{Análisis crítico}

En el ámbito educativo, las propuestas analizadas presentan distintos niveles de complejidad y enfoques pedagógicos. Entre ellas, xv6 y HobbyOS resultan especialmente adecuadas para cursos orientados a la comprensión estructural de los sistemas operativos, debido a su tamaño reducido, claridad conceptual y facilidad de ejecución. HelenOS y ToaruOS, si bien ofrecen arquitecturas más completas y modernas, se orientan mejor a contextos académicos avanzados, mientras que Visopsys constituye una alternativa intermedia que permite estudiar un sistema funcional con entorno gráfico sin llegar a la complejidad de plataformas ampliamente consolidadas.

Al comparar estas propuestas, es posible identificar patrones comunes: el uso predominante de C y ensamblador, la dependencia del emulador QEMU o herramientas similares, y la finalidad educativa o experimental que guía su diseño. No obstante, también se observan enfoques divergentes: HelenOS adopta una arquitectura microkernel multiserver, xv6 y HobbyOS se basan en kernels monolíticos simples, Visopsys integra un monolítico con GUI incorporada y ToaruOS implementa un modelo híbrido modular. Estas diferencias permiten analizar cómo cada arquitectura prioriza distintos objetivos, desde la modularidad y la tolerancia a fallos hasta la simplicidad didáctica.

A lo largo del estudio, se identifican dificultades técnicas recurrentes en todos los sistemas: la configuración de toolchains cruzados, la comprensión de los mecanismos de paginación y asignación de memoria, y la necesidad de depuración a bajo nivel. Adicionalmente, la documentación disponible varía notablemente entre proyectos, lo que influye directamente en la curva de aprendizaje. Sistemas como xv6 y ToaruOS ofrecen manuales, wikis y guías detalladas, mientras que Visopsys y particularmente HobbyOS dependen más de los archivos fuente y de documentación breve, lo cual puede constituir una barrera inicial para el análisis profundo.

Finalmente, considerando la accesibilidad técnica y el tiempo requerido para su implementación, xv6 y HobbyOS representan las opciones más factibles para un proyecto final, ya que permiten obtener resultados funcionales en poco tiempo y favorecen la comprensión de los componentes fundamentales de un sistema operativo. No obstante, elementos tomados de ToaruOS —como su enfoque modular y su servidor gráfico— y de HelenOS —particularmente su separación de servicios— pueden servir como inspiración para ampliar el alcance del proyecto sin comprometer su viabilidad. En conjunto, estas propuestas ofrecen un marco sólido para diseñar y justificar un sistema operativo académico que pueda ejecutarse y demostrarse con eficacia.

