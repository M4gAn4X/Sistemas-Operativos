\chapter{Comparación técnica}

\begin{landscape}
\begin{table}[H]
	\centering
	\caption{Comparación resumida de los sistemas operativos analizados.}
	\vspace{1em}
    \scriptsize
	\begin{tabular}{p{1.8cm}p{2cm}p{2cm}p{2cm}p{2cm}p{2cm}p{2cm}p{2cm}}
		\hline
		\textbf{Sistema} & \textbf{Arquitectura} & \textbf{Lenguaje} & \textbf{Tamaño de kernel} & \textbf{Documentación} & \textbf{Comunidad} & \textbf{Tipo de uso} & \textbf{Dificultad estimada} \\ \hline
		HelenOS & Microkernel multiserver & C, C++, ensamblador & Grande (núcleo y muchos servidores) & Alta: guías, papers y docs oficiales & Pequeña pero activa & Investigación y docencia avanzada & Alta \\
		Visopsys & Kernel monolítico modular & C y ensamblador x86 & Mediano (núcleo con GUI y VFS propio) & Media: sitio oficial y presentaciones & Muy pequeña (principalmente el autor) & Educativo y demostrativo & Media \\
		ToaruOS & Kernel híbrido modular (Misaka) & C, Kuroko, ensamblador & Mediano-grande (kernel + GUI completo) & Alta: wiki, blog y código comentado & Pequeña-moderada & Educativo y experimentación de escritorio & Media-alta \\
		xv6 & Kernel monolítico tipo UNIX V6 & C y ensamblador & Pequeño (\textasciitilde10 KLOC) & Muy alta: libro, guías de laboratorio, wiki & Amplia en el ámbito académico & Docencia universitaria de sistemas operativos & Media \\
		HobbyOS & Kernel monolítico simple & C y ensamblador x86\_64 & Muy pequeño (< 10 KLOC) & Baja-media: README y comentarios en código & Muy pequeña (proyecto individual) & Aprendizaje personal y prototipado de kernel & Baja-media \\ \hline
	\end{tabular}
\end{table}
\end{landscape}



