\chapter{Introducción}

\section{Proposito de la Investigación}


El presente documento forma parte del segundo entregable del proyecto semestral correspondiente a la asignatura de Sistemas Operativos. El propósito de este proyecto es investigar y analizar diferentes propuestas y proyectos reales de creación de sistemas operativos desde cero, identificando sus enfoques, herramientas, arquitecturas y objetos pedagógicos o técnicos.

El presente documento busca ofrecer una visión comparativa y reflexiva sobre las distintas aproximaciones existentes, con el fin de seleccionar posteriormente una base a decuada para la implementación de un sistema operativo.

\section{Criterios de selección de sistemas operativos}
La selección de las propuestas analizadas se realizó bajo criterios técnicos, pedagógicos y de accesibilidad, con el propósito de abarcar un conjunto representativo y diverso de sistemas operativos desarrollados desde cero. Los principales criterios fueron los siguientes:

\begin{itemize}
	\item \textbf{Disponibilidad del código fuente y documentación:} se priorizaron proyectos de código abierto con repositorios activos y guías técnicas verificables.
	\item \textbf{Finalidad educativa o experimental:} se consideraron  sistemas diseñados para la enseñanza, la investigación o la experimentación en el ámbito académico.
	\item \textbf{Variedad de arquitecturas y lenguajes:} se buscó incluir proyectos basados en C, C++, Rust, Assembly y otros lenguajes modernos, representando distintos paradigmas de diseño.
	\item \textbf{Diversidad arquitectónica:} se contemplaron modelos monolíticos, microkernel e híbridos, a fin de comparar enfoques estructurales.
	\item \textbf{Nivel de complejidad y accesibilidad:} los sistemas seleccionados presentan distintos grados de dificultad, desde proyectos introductorios hasta desarrollos avanzados.
\end{itemize}


\section{Metodología}

La metodología adoptada en la presente investigación es de carácter documental, comparativa y aplicada, orientada no solo a la recopilación y análisis de información técnica, sino también a la verificación práctica del funcionamiento de los sistemas operativos seleccionados. Este enfoque busca integrar el estudio teórico con la experimentación directa, fortaleciendo el aprendizaje activo y la comprensión profunda de los principios que rigen el diseño de un sistema operativo.

El desarrollo metodológico se estructuró en las siguientes fases:

\begin{enumerate}
	\item \textbf{Investigación bibliográfica y exploratoria:}
	      Se llevó a cabo una búsqueda exhaustiva de fuentes especializadas, tales como \textbf{repositorios oficiales, artículos científicos, blogs técnicos, manuales de desarrollo y documentación de proyectos open source}. Esta revisión permitió identificar iniciativas relevantes de creación de sistemas operativos desde cero.

	\item \textbf{Análisis técnico y estructural de los proyectos:}
	      En esta fase se estudió la \textbf{arquitectura interna, los componentes principales (kernel, gestor de memoria, sistema de archivos, interfaz, etc.) y los lenguajes de implementación} de cada propuesta.

	\item \textbf{Implementación y experimentación:}

	      Los proyectos seleccionados fueron \textbf{descargados, compilados y ejecutados en entornos controlados} utilizando herramientas como \textbf{QEMU, VirtualBox o VMware}, con el objetivo de observar su comportamiento real. Esta etapa permitió comprobar la \textbf{viabilidad, estabilidad y compatibilidad} de cada sistema operativo, así como analizar sus requerimientos de hardware y dependencias de compilación.

	\item \textbf{Sistematización y comparación de resultados:}
	      La información teórica y los resultados experimentales se organizaron en \textbf{tablas comparativas, gráficos y esquemas} que permiten una evaluación integral. Este proceso facilitó la identificación de \textbf{patrones comunes}, diferencias arquitectónicas y niveles de accesibilidad entre los distintos sistemas.
	\item \textbf{Análisis crítico y síntesis final:}

	      A partir de los resultados obtenidos, se elaboró una reflexión sobre la \textbf{pertinencia técnica y pedagógica} de cada propuesta, determinando cuáles podrían servir como base para el diseño e implementación del sistema operativo propio del grupo.
	\item \textbf{Elaboración del informe:}
\end{enumerate}
