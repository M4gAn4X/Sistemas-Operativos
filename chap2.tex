\section{Propuestas Analizadas}

% =========================
% Propuesta 1: HelenOS
% =========================
\subsection{HelenOS}

\subsubsection{Nombre del proyecto o sistema operativo}
HelenOS es un sistema operativo multiserver basado en un microkernel, desarrollado completamente desde cero y sin dependencia de Unix. Su diseño busca la modularidad, la portabilidad y la tolerancia a fallos, orientado principalmente a la investigación académica y la enseñanza de arquitectura de sistemas operativos \cite{helenos_about}.

\subsubsection{Enlace al repositorio y/o documentación oficial}
Repositorio oficial: \url{https://github.com/HelenOS/helenos}.

Sitio web del proyecto: \url{https://www.helenos.org}.

Documentación técnica (PDF): \url{https://www.helenos.org/doc/prjdoc.pdf}

\subsubsection{Objetivo del proyecto (educativo, experimental, comercial)}
El objetivo de HelenOS es principalmente \textbf{educativo y experimental}. Su propósito es servir como plataforma de estudio de sistemas operativos modernos, promoviendo un enfoque modular donde cada servicio (sistema de archivos, red, controladores o interfaz gráfica) se implementa como un servidor independiente en espacio de usuario. Este diseño fomenta el análisis de conceptos avanzados como la comunicación entre procesos (IPC) y la separación entre el núcleo y los servicios de usuario \cite{helenos_about}.

\subsubsection{Lenguaje(s) de implementación}
El proyecto está desarrollado principalmente en el lenguaje \textbf{C}, con partes críticas escritas en \textbf{ensamblador} (para el arranque del sistema y la gestión de interrupciones). Se mantiene compatibilidad con los estándares \textbf{C11} y \textbf{C++14}, facilitando la portabilidad y la integración de nuevos componentes \cite{helenos_prjdoc}.

\subsubsection{Arquitectura del sistema (monolítica, microkernel, etc.)}
HelenOS adopta una arquitectura \textbf{microkernel multiserver}. El microkernel proporciona únicamente los servicios esenciales —planificación, gestión de memoria, interrupciones e IPC— mientras que todos los demás servicios (como el sistema de archivos, la red o la interfaz gráfica) se ejecutan como procesos de usuario independientes que se comunican mediante paso de mensajes (\emph{message passing}) \cite{helenos_prjdoc}.

\subsubsection{Componentes implementados (procesos, memoria, archivos, etc.)}
\begin{itemize}
	\item \textbf{Gestión de procesos e hilos:} Soporta multitarea con planificación por prioridades y aislamiento entre procesos. El modelo de ejecución permite múltiples hilos por proceso y comunicación sincronizada entre ellos mediante IPC.

	\item \textbf{Gestión de memoria:} Implementa un sistema de memoria virtual con paginación y protección por espacio de direcciones. El kernel se encarga de la asignación básica y los servidores de usuario gestionan asignadores dinámicos (\textit{slab allocators}) y regiones compartidas.

	\item \textbf{Sistema de archivos (VFS):} Se basa en una arquitectura de VFS (Virtual File System) donde cada tipo de sistema de archivos (por ejemplo, FAT, EXT2) se implementa como un servidor separado en espacio de usuario. Esto permite montar y desmontar sistemas de archivos sin reiniciar el núcleo.

	\item \textbf{Dispositivos y controladores:} Los \textit{drivers} se ejecutan fuera del núcleo (user-space drivers), lo que evita que los errores en controladores afecten la estabilidad general del sistema. Hay soporte para dispositivos de almacenamiento, red, USB, video, teclado y mouse.

	\item \textbf{Sistema de red:} Incluye una pila TCP/IP modular con soporte para IPv4, interfaces Ethernet virtuales, y utilidades de diagnóstico como \texttt{ping}, \texttt{netstat}, \texttt{inet list-addr} y \texttt{arp}.

	\item \textbf{Sistema gráfico:} Dispone de un entorno gráfico de ventanas (GUI) que corre completamente en espacio de usuario. Proporciona aplicaciones nativas como \textit{Terminal}, \textit{Text Editor}, \textit{Navigator}, \textit{Calculator}, \textit{GFX Demo} y \textit{Tetris}. El gestor de ventanas se comunica con el servidor gráfico a través del protocolo \texttt{display-server}.

	\item \textbf{Comunicación e IPC:} El núcleo ofrece un sistema de comunicación entre procesos mediante intercambio de mensajes (\textit{message passing}) orientado a puertos. Este mecanismo es la base de toda interacción entre servidores y aplicaciones.

	\item \textbf{Seguridad y aislamiento:} Cada componente ejecuta con privilegios mínimos. Si un servidor falla, el kernel lo aísla y otros componentes pueden seguir ejecutándose, lo que mejora la tolerancia a fallos.

	\item \textbf{Portabilidad:} El código fuente está diseñado para ser independiente de la arquitectura. Actualmente soporta múltiples plataformas: \texttt{ia32}, \texttt{amd64}, \texttt{arm32}, \texttt{mips32}, \texttt{ppc32}, \texttt{sparc64} e incluso \texttt{ia64} (Itanium).
\end{itemize}

\subsubsection{Herramientas utilizadas (compiladores, emuladores, etc.)}

El proceso de compilación y prueba de HelenOS requiere la preparación de un entorno de desarrollo cruzado (\emph{cross-compiling}) que permite generar imágenes del sistema para múltiples arquitecturas. A continuación se detallan las principales herramientas utilizadas:

\begin{itemize}
	\item \textbf{Toolchain cruzado:} Se genera mediante el script \texttt{tools/toolchain.sh}, que compila una cadena completa de herramientas \texttt{gcc}, \texttt{binutils} y \texttt{gdb} adaptadas para la arquitectura destino (por ejemplo, \texttt{ia64-helenos-gcc}). El entorno de compilación se instala por defecto en el directorio \texttt{/usr/local/cross} y puede personalizarse mediante la variable \texttt{CROSS\_PREFIX}.

	\item \textbf{Meson y Ninja:} HelenOS utiliza el sistema de construcción \textbf{Meson} junto al generador \textbf{Ninja}, lo que permite una configuración modular y compilaciones rápidas e incrementales.

	\item \textbf{Dependencias del entorno:} Para distribuciones basadas en Debian/Ubuntu, se necesitan los paquetes \texttt{build-essential}, \texttt{wget}, \texttt{texinfo}, \texttt{flex}, \texttt{bison}, \texttt{dialog}, \texttt{python3-yaml} y \texttt{genisoimage}.

	\item \textbf{Configuración del proyecto:} El script \texttt{configure.sh} permite definir perfiles preestablecidos de compilación. Por ejemplo, para la arquitectura \texttt{amd64}:
	      \begin{verbatim}
  git clone https://github.com/HelenOS/helenos.git
  mkdir -p build/amd64
  cd build/amd64
  ../../helenos/configure.sh amd64
  ninja
  ninja image_path
  \end{verbatim}

	\item \textbf{Emulación y prueba:} HelenOS puede ejecutarse en \textbf{QEMU} o \textbf{VirtualBox}. Para iniciar el sistema en QEMU se recomienda:
	      \begin{verbatim}
  qemu-system-x86_64 -m 512 -cdrom image.iso \
  -usb -device usb-tablet -serial mon:stdio -display sdl
  \end{verbatim}
\end{itemize}

Además, el repositorio incluye scripts auxiliares en \texttt{tools/ew.py} que automatizan el proceso de arranque en QEMU, y configuraciones predefinidas en \texttt{tools/conf} para ajustar parámetros de memoria o dispositivos virtuales.

\begin{figure}[H]
  \centering
  \includegraphics[width=1\linewidth]{figures/helenos_toolchain.png}
  \caption{Elaboración Propia. Compilación cruzada del toolchain.sh}
\end{figure}

\subsubsection{Nivel de complejidad y accesibilidad}
El nivel de complejidad de HelenOS es alto, ya que su comprensión requiere conocimientos de compilación cruzada, IPC y gestión de memoria. Sin embargo, su estructura modular y bien documentada lo convierte en una excelente herramienta didáctica donde se pueden analizar de manera aislada servicios específicos como el sistema de archivos o el entorno gráfico \cite{helenos_docs}.

\begin{table}[!htbp]
	\centering
	\caption{Ficha técnica resumida de HelenOS.}
	\vspace{1em}
	\begin{tabular}{p{4cm}p{10cm}}
		\hline
		\textbf{Diseño}            & Microkernel multiserver modular                                     \\ \hline
		\textbf{Lenguajes}         & C (principal) y ensamblador (arranque y bajo nivel)                 \\
		\textbf{Arquitecturas}     & ia32, amd64, arm32, mips32, ppc32, sparc64                          \\
		\textbf{Componentes clave} & Kernel mínimo, VFS, red, GUI, drivers en espacio de usuario         \\
		\textbf{Herramientas}      & GCC (toolchain cruzado), Meson, Ninja, QEMU/VirtualBox              \\
		\textbf{Uso educativo}     & Alto: análisis modular de componentes y comunicación entre procesos \\
		\hline
	\end{tabular}\\
	\vspace{1mm}
	\textit{Nota.} Adaptado de “About HelenOS” (HelenOS Project, 2024)
\end{table}


